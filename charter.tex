\documentclass[
11pt, % The default document font size, options: 10pt, 11pt, 12pt
codirector, % Uncomment to add a codirector to the title page
]{charter} 




% El títulos de la memoria, se usa en la carátula y se puede usar el cualquier lugar del documento con el comando \ttitle
\titulo{Firmware de comunicaciones para un tren motor} 

% Nombre del posgrado, se usa en la carátula y se puede usar el cualquier lugar del documento con el comando \degreename
\posgrado{Carrera de Especialización en Sistemas Embebidos} 
%\posgrado{Carrera de Especialización en Internet de las Cosas} 
%\posgrado{Carrera de Especialización en Intelegencia Artificial}
%\posgrado{Maestría en Sistemas Embebidos} 
%\posgrado{Maestría en Internet de las cosas}

% Tu nombre, se puede usar el cualquier lugar del documento con el comando \authorname
\autor{Marcos Raul Dominguez Shocron} 

% El nombre del director y co-director, se puede usar el cualquier lugar del documento con el comando \supname y \cosupname y \pertesupname y \pertecosupname
\director{A definir}
\pertenenciaDirector{pertenencia} 
% FIXME:NO IMPLEMENTADO EL CODIRECTOR ni su pertenencia
\codirector{John Doe} % para que aparezca en la portada se debe descomentar la opción codirector en el documentclass
\pertenenciaCoDirector{FIUBA}

% Nombre del cliente, quien va a aprobar los resultados del proyecto, se puede usar con el comando \clientename y \empclientename
\cliente{Guillermo Gebhart}
\empresaCliente{Voltu Motors}

% Nombre y pertenencia de los jurados, se pueden usar el cualquier lugar del documento con el comando \jurunoname, \jurdosname y \jurtresname y \perteunoname, \pertedosname y \pertetresname.
\juradoUno{Nombre y Apellido (1)}
\pertenenciaJurUno{pertenencia (1)} 
\juradoDos{Nombre y Apellido (2)}
\pertenenciaJurDos{pertenencia (2)}
\juradoTres{Nombre y Apellido (3)}
\pertenenciaJurTres{pertenencia (3)}
 
\fechaINICIO{08 de marzo de 2022}		%Fecha de inicio de la cursada de GdP \fechaInicioName
\fechaFINALPlan{19 de abril de 2022} 	%Fecha de final de cursada de GdP
\fechaFINALTrabajo{a definir}	%Fecha de defensa pública del trabajo final


\begin{document}

\maketitle
\thispagestyle{empty}
\pagebreak


\thispagestyle{empty}
{\setlength{\parskip}{0pt}
	\tableofcontents{}
}
\pagebreak


\section*{Registros de cambio}
\label{sec:registro}


\begin{table}[ht]
	\label{tab:registro}
	\centering
	\begin{tabularx}{\linewidth}{@{}|c|X|c|@{}}
		\hline
		\rowcolor[HTML]{C0C0C0}
		Revisión & \multicolumn{1}{c|}{\cellcolor[HTML]{C0C0C0}Detalles de los cambios realizados} & Fecha            \\ \hline
		0        & Creación del documento                                                          & \fechaInicioName \\ \hline
		1        & Se completa hasta el punto 5 inclusive                                          & 13/03/2022       \\ \hline
		2        & Se completa hasta el punto 9 inclusive                                          & 20/03/2022       \\ \hline
		3        & Se completa hasta el punto 12 inclusive                                         & 29/03/2022       \\ \hline
		%		  Se puede agregar algo más \newline
		%		  En distintas líneas \newline
		%		  Así                                                    & dd/mm/aaaa \\ \hline
		%3      & Se completa hasta el punto 11 inclusive                & dd/mm/aaaa \\ \hline
		%4      & Se completa el plan	                                 & dd/mm/aaaa \\ \hline
	\end{tabularx}
\end{table}

\pagebreak



\section*{Acta de constitución del proyecto}
\label{sec:acta}

\begin{flushright}
	Buenos Aires, \fechaInicioName
\end{flushright}

\vspace{2cm}

Por medio de la presente se acuerda con el Ing. \authorname\hspace{1px} que su Trabajo Final de la \degreename\hspace{1px} se titulará ``\ttitle'', consistirá esencialmente en la implementación del firmware sobre la placa de comunicaciones de un Tren Motor, y tendrá un presupuesto preliminar estimado de 596 hs de trabajo y \$3447132, con fecha de inicio \fechaInicioName\hspace{1px} y fecha de presentación pública \fechaFinalName.

Se adjunta a esta acta la planificación inicial.

\vfill

% Esta parte se construye sola con la información que hayan cargado en el preámbulo del documento y no debe modificarla
\begin{table}[ht]
	\centering
	\begin{tabular}{ccc}
		\begin{tabular}[c]{@{}c@{}}Ariel Lutenberg \\ Director posgrado FIUBA\end{tabular} & \hspace{2cm} & \begin{tabular}[c]{@{}c@{}}\clientename \\ \empclientename \end{tabular} \vspace{2.5cm} \\
		\multicolumn{3}{c}{\begin{tabular}[c]{@{}c@{}} \supname \\ Director del Trabajo Final\end{tabular}} \vspace{2.5cm}                        \\
		%\begin{tabular}[c]{@{}c@{}}\jurunoname \\ Jurado del Trabajo Final\end{tabular}     &  & \begin{tabular}[c]{@{}c@{}}\jurdosname\\ Jurado del Trabajo Final\end{tabular}  \vspace{2.5cm}  \\
		%\multicolumn{3}{c}{\begin{tabular}[c]{@{}c@{}} \jurtresname\\ Jurado del Trabajo Final\end{tabular}} \vspace{.5cm}                                                                     
	\end{tabular}
\end{table}




\section{1. Descripción técnica-conceptual del proyecto a realizar}
\label{sec:descripcion}


Un Power Train o Tren Motor es el sistema encargado de impulsar los vehículos eléctricos. Estos convierten la energía eléctrica almacenada en baterías a motriz alimentando un motor eléctrico.

El tren motor es uno de los dos productos principales que desarrolla y comercializa la empresa Voltu Motors. Este Tren Motor puede adaptarse a diferentes tipos de vehículos, actualmente se han instalado en motos, cuatriciclos, colectivos y camiones.

El dispositivo consta de un pack de baterías, una unidad de control, el motor. La unidad de control de Voltu Motors utiliza dos placas microcontroladas para resolver el hardware.
La primera de ellas es un DSP que realiza el control para mover el motor y cargar las baterías . La segunda es la placa de comunicaciones que cumple la función de interactuar con el exterior de la unidad de control.

El usuario del vehículo puede controlar el Tren Motor mediante una interfaz gráfica, la cual se encuentra en el tablero del vehículo. Este tablero es una tablet que se comunica con la unidad de control mediante protocolo USB.

La placa de comunicaciones posee:

\begin{itemize}
	\item Entradas y salidas digitales para el manejo de periféricos como luces, ventiladores, etc.
	\item Interfaz USB para comunicarse con la tablet del usuario.
	\item Interfaz UART para comunicarse con la placa de control, una con fines de \textit{debug} y otra para comunicarse con el BMS (\textit{Batery Management System}).
	\item Interfaz para gestionar el conexionado de un EVSE (\textit{Electric Vehicle Supply Equipment}).
\end{itemize}

Las características listadas anteriormente convierten a la placa de comunicaciones en la encargada de gestionar el estado de la unidad de control. Esta utiliza la información que recibe de todas sus interfaces para la toma de decisiones.

Actualmente, existe una versión funcional con un hardware distinto en donde la placa de control gestiona gran parte de las tareas que se proponen para la nueva placa de comunicaciones.

La nueva implementación permitirá reducir las tareas del DSP a los algoritmos de control del motor y la carga, y mejora así la seguridad del sistema. Además, esta configuración incorpora nuevas características, como el manejo de distintos tipos de EVSE y la posibilidad de definir fuera de tiempo de compilación características del tren motor que se adapten a distintos tipos de vehículos.

%\vspace{25px}


En la Figura \ref{fig:diagBloques} se presenta el diagrama en bloques del sistema. Se puede observar como la placa de comunicaciones interactúa con la Tablet, el DSP y el BMS, mientras que la placa de control se encarga de controlar el flujo de energía para la carga de la batería o marcha del motor. Adicionalmente, la placa de comunicaciones gestiona las entradas y salidas con los periféricos del vehículo.

Debido a que la nueva versión del tren motor está planificada con este hardware, el desarrollo de este proyecto es fundamental para su lanzamiento.

\begin{figure}[htpb]
	\centering
	\includegraphics[width=.9\textwidth]{./Figuras/EsquematicoPT.pdf}
	\caption{Diagrama en bloques del sistema}
	\label{fig:diagBloques}
\end{figure}

\vspace{25px}

\section{2. Identificación y análisis de los interesados}
\label{sec:interesados}

\begin{table}[ht]
	%\caption{Identificación de los interesados}
	%\label{tab:interesados}
	\begin{tabularx}{\linewidth}{@{}|l|X|X|l|@{}}
		\hline
		\rowcolor[HTML]{C0C0C0}
		Rol           & Nombre y Apellido & Organización    & Puesto                      \\ \hline
		Auspiciante   & \clientename      & \empclientename & CEO y Fundador              \\ \hline
		Cliente       & \clientename      & \empclientename & CEO y Fundador              \\ \hline
		Impulsor      & Luciano Vittori   & \empclientename & Lider de Control y Firmware \\ \hline
		Responsable   & \authorname       & FIUBA           & Alumno                      \\ \hline
		Orientador    & \supname          & \pertesupname   & Director Trabajo final      \\ \hline
		Usuario final & Gonzalo Cuenca    & \empclientename & Testing y Calidad           \\ \hline
	\end{tabularx}
\end{table}





\section{3. Propósito del proyecto}
\label{sec:proposito}

El propósito de este proyecto es desarrollar el firmware de la nueva placa de comunicaciones. Este firmware debe cubrir las características de la versión anterior y además contemplar el manejo del BMS, EVSE y periféricos.
La nueva versión también será la encargada de realizar el control de temperatura del sistema.

\section{4. Alcance del proyecto}
\label{sec:alcance}

En el presente proyecto se diseñará, desarrollará e implementará el firmware de la placa de comunicaciones hasta que sea funcional para un vehículo que utilice una sola unidad de tren motor.
Esto incluye la implementación de las siguientes características:
\begin{itemize}
	\item Manejo de entradas y salidas digitales.
	\item Comunicación USB con el protocolo actual para la comunicación con la Tablet.
	\item Comunicación con placa de control, transmisión de datos y recepción de datos.
	\item Gestión del estado del vehículo.
	\item Administración de la batería (definir cuando cargar y balancear las celdas).
	\item Control de temperatura del sistema.
\end{itemize}

Este proyecto no incluye el diseño del hardware debido a que ya está preestablecido. El montaje del sistema en un vehículo particular tampoco está dentro del alcance del proyecto.


\section{5. Supuestos del proyecto}
\label{sec:supuestos}

Para el desarrollo del presente proyecto se supone que:

\begin{itemize}
	\item Se dispone del hardware de la nueva placa de comunicaciones.
	\item Se dispone de un banco de pruebas con el sistema completo para realizar pruebas de inmunidad al ruido del firmware.
	\item Se dispone de un EVSE para las pruebas de carga.
	\item Se dispone de módulos de batería con placas de BMS montada para desarrollar las estrategias de administración de batería.
	\item Se dispone de un software de depuración USB para relevar la información del sistema.
	\item Se dispone de alimentación trifásica para realizar los ciclos de carga.
\end{itemize}


\section{6. Requerimientos}
\label{sec:requerimientos}

\begin{enumerate}
	\item Interfaces
	      \begin{enumerate}
		      \item El sistema debe poder comunicarse con el tablero con protocolo USB.
		      \item El sistema debe poseer un botón de encendido
		      \item El sistema debe poder actuar sobre los periféricos de refrigeración.
		      \item El sistema debe responder a las señales digitales de los pulsadores y switches del vehículo. El tiempo de respuesta debe ser instantáneo para la percepción del usuario final (menor a 100 ms).
		      \item El sistema debe comunicarse con el BMS.
		      \item El sistema debe comunicarse con la placa de control. Esta comunicación debe ser redundante y robusta.
		      \item El sistema debe comunicarse con el EVSE.
	      \end{enumerate}
	\item Requerimientos funcionales
	      \begin{enumerate}
		      \item Comunicación con el tablero.
		            \begin{enumerate}
			            \item La taza de actualización de datos debe ser de 10 ms.
			            \item Se debe transmitir al tablero toda la información reportada por la placa de control.
			            \item El estado de las entradas y salidas digitales debe ser reportado al tablero.
			            \item Las temperaturas de la unidad de control, batería y motor deben ser reportadas al tablero.
			            \item El sistema puede ser configurado mediante el usuario a través del tablero.
		            \end{enumerate}
		      \item Entradas digitales
		            \begin{enumerate}
			            \item El sistema debe poder recibir señales digitales de los pulsadores y switches del vehículo.
			            \item Las señales deben ser remapeables por cada vehículo, es decir, disociar el pin físico de la funcionalidad específica.
			            \item La activación de las entradas digitales (con tensión o masa) deben ser configurables fuera del tiempo de compilación por personal de la empresa.
		            \end{enumerate}
		      \item BMS
		            \begin{enumerate}
			            \item El sistema debe conocer el estado de todas las celdas.
			            \item El sistema debe conocer las temperaturas en el interior de todos los módulos de batería.
			            \item El sistema debe ser capaz de mantener las baterías balanceadas. Un máximo de 50 mV entre celdas.
			            \item El sistema debe ser capaz de conectarse a distintas configuraciones de baterías según los parámetros del vehículo. Cada vehículo posee distinta cantidad de módulos y distintos modelos de módulos.
		            \end{enumerate}
		      \item Control de temperaturas
		            \begin{enumerate}
			            \item El sistema debe conocer la temperatura de la unidad de control.
			            \item El sistema debe conocer la temperatura de la batería.
			            \item El sistema debe conocer la temperatura del motor.
			            \item El sistema debe accionar los actuadores de refrigeración cuando la temperatura de algún componente exceda el 60\% de la temperatura de falla.
			            \item Las bombas de circulación de liquido deben accionarse siempre que el vehículo se encuentre en marcha o carga.
		            \end{enumerate}
		      \item Carga con EVSE
		            \begin{enumerate}
			            \item El sistema debe poder conectarse con un EVSE Mennekes.
			            \item El sistema debe interpretar el limite de corriente que informa el EVSE para configurar el valor de la corriente de carga.
			            \item El sistema debe conocer la temperatura del EVSE.
			            \item El sistema debe disparar la orden de carga a la placa de control cuando el EVSE esté conectado y listo para cargar.
			            \item La placa de comunicaciones debe sacar al vehículo de marcha si detecta la conexión de un EVSE.
		            \end{enumerate}
		      \item Comunicación con placa de control
		            \begin{enumerate}
			            \item La placa de comunicaciones debe indicar a la placa de control el estado objetivo (marcha, carga o reposo).
			            \item La placa de comunicaciones debe indicar a la placa de control la dirección de avance (Directa o Reversa).
			            \item La placa de comunicaciones debe informar la tensión máxima y mínima de las celdas a la placa de control.
			            \item La placa de comunicaciones debe ser capaz de recibir datos de la placa de control cada 10 ms.
		            \end{enumerate}
	      \end{enumerate}
	\item Requerimientos de robustez
	      \begin{enumerate}
		      \item El sistema debe ser capaz de mantener sus comunicaciones en condiciones ruidosas (aceleraciones fuertes o cargas rápidas).
	      \end{enumerate}
	\item Requerimientos de testing
	      \begin{enumerate}
		      \item El sistema debe ser poseer un modo de servicio técnico.
		      \item El modo de servicio técnico debe permitir probar los actuadores del sistema de forma independiente.
		      \item El modo de servicio técnico debe ser vía PC mediante el mismo USB del del tablero.
		      \item El modo de servicio técnico debe permitir la configuración de los parámetros del sistema.
	      \end{enumerate}
\end{enumerate}


\section{7. Historias de usuarios (\textit{Product backlog})}
\label{sec:backlog}

En esta sección se incluyen las historias de usuarios y su ponderación (\textit{story points})

``Como usuario quiero poder conectar el cargador y realizar una carga del vehículo."

Dificultad: alta (5) - Implica muchas horas de estudio, diseño e implementación para lograr una coordinación entre el EVSE, placa de comunicaciones, placa de control y BMS.

Complejidad: alta (5) - Realizar un diseño seguro y bien coordinado entre los sistemas es fundamental debido a que se utilizan potencias de varios KW.

Riesgo: alto (5) - Durante los ensayos de esta funcionalidad existe el riesgo de electrochoque e incendio de baterías.

\textit{story points}: 13
(5 + 5 + 5 = 15 -- 13 es el valor más cercano en Fibonacci)

``Como desarrollador quiero poder validar el funcionamiento de los actuadores de refrigeración."

Dificultad: alta (5) - Implica el desarrollo de un sistema de servicio técnico que permita interpretar comandos por un protocolo USB.

Complejidad: media (3) - La creación del sistema y protocolo de comunicación implica una complejidad elevada. Pero la conmutación de los actuadores una vez que se interpreta el comando no lo es.

Riesgo: bajo (1) - No hay riesgos importantes al utilizar esta funcionalidad.

\textit{story points}: 13
(5 + 3 + 1 = 9 -- 8 es el valor más cercano en Fibonacci)

``Como desarrollador quiero poder configurar los parámetros de un tren motor conectándome con una PC al equipo."

Dificultad: alta (4) - Implica muchas horas de diseño de firmware para parametrizar el vehículo con una lista acotada de parámetros.

Complejidad: baja (1) - La dependencia de valores y condiciones según valores de parámetros no conlleva una gran complejidad.

Riesgo: bajo (1) - El uso de una pc mediante USB para modificar valores del equipo no es una actividad riesgosa.

\textit{story points}: 8
(4 + 1 + 1 = 6 -- 5 es el valor más cercano en Fibonacci)

``Como usuario quiero encender el vehículo y marcharlo"

Dificultad: alta (5) - Implica muchas horas de diseño de firmware para coordinar las instrucciones del usuario con la placa de control mientras se monitorea la integridad del sistema para evitar accidentes.

Complejidad: alta (5) - Contemplar las totalidad de la situaciones para garantizar la seguridad del usuario durante la marcha es una actividad muy compleja.

Riesgo: alto (5) - Una falla en es sistema no contemplado puede tener como consecuencia una fatalidad.

\textit{story points}: 8
(5 + 5 + 5 = 15 -- 13 es el valor más cercano en Fibonacci)

``Como usuario quiero poder encender las luces de mi vehículo."

Dificultad: baja (1) - Implica atender entradas digitales y activar salidas digitales.

Complejidad: baja (1) - El manejo de entradas y salidas digitales de baja complejidad en el desarrollo de firmware.

Riesgo: bajo (1) - No existen grandes riesgos encendiendo y apagando luces que funcionan a 12 V.

\textit{story points}: 8
(1 + 1 + 1 = 3 -- 3 es el valor más cercano en Fibonacci)

\section{8. Entregables principales del proyecto}
\label{sec:entregables}

Los entregables del proyecto son:

\begin{itemize}
	\item Diagrama del firmware
	\item Código fuente del firmware
	\item Manual de uso para el sistema de servicio técnico
	\item Informe final
\end{itemize}

\section{9. Desglose del trabajo en tareas}
\label{sec:wbs}


\begin{enumerate}
	\item Comunicación con el tablero (92 horas).
	      \begin{enumerate}
		      \item Implementación de capa USB e implementación del protocolo de comunicación (40 horas).
		      \item Implementación del empaquetado de la información de la placa de control para su retransmisión (16 horas).
		      \item Implementación del reporte de entradas y salidas digitales al tablero (8 horas).
		      \item Implementación del empaquetado de las temperaturas y envío al tablero (8 horas).
		      \item Implementación de protocolo para recibir instrucciones desde el tablero (16 horas).
		      \item Validación de la comunicación a 10 ms (4 horas).
	      \end{enumerate}
	\item Entradas digitales (40 horas).
	      \begin{enumerate}
		      \item Implementación de capa de atención a entradas digitales con abstracción entre la función y el pin físico (16 horas).
		      \item Implementación de capa de activación de salidas digitales con abstracción entre la función y el pin físico (16 horas).
		      \item Implementación de la lógica para configurar el mecanismo de activación de las entradas digitales (con tensión o masa) (8 horas).
	      \end{enumerate}
	\item BMS (80 horas).
	      \begin{enumerate}
		      \item Implementar la configuración de los parámetros de los BMS (16 horas).
		      \item Implementación de la inicialización del sistema de BMS según los parámetros de configuración (32 horas).
		      \item Implementar las lecturas de tensiones de celdas (8 horas).
		      \item Implementar las lecturas de temperaturas de módulos (8 horas).
		      \item Implementar lógica y mecanismo de balanceo de celdas (16 horas).
	      \end{enumerate}
	\item Control de temperaturas (56 horas)
	      \begin{enumerate}
		      \item Activación de entradas analógicas para la medición de temperatura de motor (16 horas).
		      \item Sustracción de la temperaturas de la unidad de control informada por la placa de control (8 horas).
		      \item Implementación de lógica de control de temperaturas según los limites establecidos (histéresis entre el 45\% y 55\% de las temperaturas de falla) (16 horas).
		      \item Validación del correcto funcionamiento del sistema (16 horas).
	      \end{enumerate}
	\item Carga con EVSE (80 horas).
	      \begin{enumerate}
		      \item Implementar la detección de conexión del EVSE (16 horas).
		      \item Implementación de la decodificación del PWM para conocer la corriente máxima de la carga (16 horas).
		      \item Implementación de la lógica de carga y coordinación entre el BMS, EVSE y placa de control (32 horas).
		      \item Implementación de la salida de marcha ante el conexionado del EVSE (8 horas).
		      \item Validación del sistema con una carga a 16 A por fase (8 horas).
	      \end{enumerate}
	\item Comunicación con placa de control (56 horas).
	      \begin{enumerate}
		      \item Implementación de la recepción de datos de la placa de control vía UART (32 horas).
		      \item Implementación del envío de datos de la placa de control vía UART (16 horas).
		      \item Validación de la integridad de los datos a la tasa exigida por los requerimientos (10 ms) (8 horas).
	      \end{enumerate}
	\item Modo servicio técnico (72 horas).
	      \begin{enumerate}
		      \item Implementación del modo de servicio técnico (16 horas).
		      \item Creación e implementación de comandos de servicio técnico (16 horas).
		      \item Prueba de los comandos de servicio técnico (8 horas).
		      \item Redacción de manual de uso para el sistema de servicio técnico (32 horas).
	      \end{enumerate}
	\item Informe final (120 horas).
	      \begin{enumerate}
		      \item Redacción del marco teórico (40 horas).
		      \item Realización de diagramas del firmware (40 horas).
		      \item Redacción de resultados (40 horas).
	      \end{enumerate}
\end{enumerate}


Cantidad total de horas: 596 horas.




\section{10. Diagrama de Activity On Node}
\label{sec:AoN}

\begin{figure}[htpb]
	\centering
	\includegraphics[width=.8\textwidth]{./Figuras/AoN.pdf}
	\caption{Diagrama en \textit{Activity on Node}}
	\label{fig:AoN}
\end{figure}

\section{11. Diagrama de Gantt}
\label{sec:gantt}


\begin{landscape}
	\begin{figure}[htpb]
		\centering
		\includegraphics[height=.78\textheight]{./Figuras/Gantt.png}
		\caption{Diagrama de Gantt}
		\label{fig:diagGantt}
	\end{figure}

\end{landscape}



\section{12. Presupuesto detallado del proyecto}
\label{sec:presupuesto}

\begin{table}[htpb]
	\centering
	\begin{tabularx}{\linewidth}{@{}|X|c|r|r|@{}}
		\hline
		\rowcolor[HTML]{C0C0C0}
		\multicolumn{4}{|c|}{\cellcolor[HTML]{C0C0C0}COSTOS DIRECTOS}   \\ \hline
		\rowcolor[HTML]{C0C0C0}
		Descripción                                                 &
		\multicolumn{1}{c|}{\cellcolor[HTML]{C0C0C0}Cantidad}       &
		\multicolumn{1}{c|}{\cellcolor[HTML]{C0C0C0}Valor unitario} &
		\multicolumn{1}{c|}{\cellcolor[HTML]{C0C0C0}Valor total}        \\ \hline
		Placa de comunicaciones                                             &
		\multicolumn{1}{c|}{2}                                      &
		\multicolumn{1}{c|}{10000}                                  &
		\multicolumn{1}{c|}{20000}                                      \\ \hline
		Placa de control                                        &
		\multicolumn{1}{c|}{2}                                      &
		\multicolumn{1}{c|}{15000}                                  &
		\multicolumn{1}{c|}{30000}                                      \\ \hline
		EVSE                      &
		\multicolumn{1}{c|}{1}                                      &
		\multicolumn{1}{c|}{60000}                                   &
		\multicolumn{1}{c|}{60000}                                       \\ \hline
		Modulo de baterías con BMS                     &
		\multicolumn{1}{c|}{4}                                      &
		\multicolumn{1}{c|}{410400}                                   &
		\multicolumn{1}{c|}{1641640}                                       \\ \hline
		Honorarios de desarrollador                                 &
		\multicolumn{1}{c|}{600}                                    &
		\multicolumn{1}{c|}{1500}                                   &
		\multicolumn{1}{c|}{900000}                                     \\ \hline
		\multicolumn{3}{|c|}{SUBTOTAL}                              &
		\multicolumn{1}{c|}{2651640}                                     \\ \hline
		\rowcolor[HTML]{C0C0C0}
		\multicolumn{4}{|c|}{\cellcolor[HTML]{C0C0C0}COSTOS INDIRECTOS} \\ \hline
		\rowcolor[HTML]{C0C0C0}
		Descripción                                                 &
		\multicolumn{1}{c|}{\cellcolor[HTML]{C0C0C0}Cantidad}       &
		\multicolumn{1}{c|}{\cellcolor[HTML]{C0C0C0}Valor unitario} &
		\multicolumn{1}{c|}{\cellcolor[HTML]{C0C0C0}Valor total}        \\ \hline
		\multicolumn{1}{|l|}{30\% de los costos directos}           &
		\multicolumn{1}{c|}{1}                                      &
		\multicolumn{1}{c|}{795492}                                 &
		\multicolumn{1}{c|}{795492}                                     \\ \hline
		\multicolumn{3}{|c|}{SUBTOTAL}                              &
		\multicolumn{1}{c|}{294300}                                     \\ \hline
		\rowcolor[HTML]{C0C0C0}
		\multicolumn{3}{|c|}{TOTAL}                                 &
		\multicolumn{1}{c|}{3447132}                                    \\ \hline
	\end{tabularx}%
\end{table}


\section{13. Gestión de riesgos}
\label{sec:riesgos}

\begin{consigna}{red}
	a) Identificación de los riesgos (al menos cinco) y estimación de sus consecuencias:

	Riesgo 1: detallar el riesgo (riesgo es algo que si ocurre altera los planes previstos de forma negativa)
	\begin{itemize}
		\item Severidad (S): mientras más severo, más alto es el número (usar números del 1 al 10).\\
		      Justificar el motivo por el cual se asigna determinado número de severidad (S).
		\item Probabilidad de ocurrencia (O): mientras más probable, más alto es el número (usar del 1 al 10).\\
		      Justificar el motivo por el cual se asigna determinado número de (O).
	\end{itemize}

	Riesgo 2:
	\begin{itemize}
		\item Severidad (S):
		\item Ocurrencia (O):
	\end{itemize}

	Riesgo 3:
	\begin{itemize}
		\item Severidad (S):
		\item Ocurrencia (O):
	\end{itemize}


	b) Tabla de gestión de riesgos:      (El RPN se calcula como RPN=SxO)

	\begin{table}[htpb]
		\centering
		\begin{tabularx}{\linewidth}{@{}|X|c|c|c|c|c|c|@{}}
			\hline
			\rowcolor[HTML]{C0C0C0}
			Riesgo & S & O & RPN & S* & O* & RPN* \\ \hline
			       &   &   &     &    &    &      \\ \hline
			       &   &   &     &    &    &      \\ \hline
			       &   &   &     &    &    &      \\ \hline
			       &   &   &     &    &    &      \\ \hline
			       &   &   &     &    &    &      \\ \hline
		\end{tabularx}%
	\end{table}

	Criterio adoptado:
	Se tomarán medidas de mitigación en los riesgos cuyos números de RPN sean mayores a...

	Nota: los valores marcados con (*) en la tabla corresponden luego de haber aplicado la mitigación.

	c) Plan de mitigación de los riesgos que originalmente excedían el RPN máximo establecido:

	Riesgo 1: plan de mitigación (si por el RPN fuera necesario elaborar un plan de mitigación).
	Nueva asignación de S y O, con su respectiva justificación:
	- Severidad (S): mientras más severo, más alto es el número (usar números del 1 al 10).
	Justificar el motivo por el cual se asigna determinado número de severidad (S).
	- Probabilidad de ocurrencia (O): mientras más probable, más alto es el número (usar del 1 al 10).
	Justificar el motivo por el cual se asigna determinado número de (O).

	Riesgo 2: plan de mitigación (si por el RPN fuera necesario elaborar un plan de mitigación).

	Riesgo 3: plan de mitigación (si por el RPN fuera necesario elaborar un plan de mitigación).

\end{consigna}


\section{14. Gestión de la calidad}
\label{sec:calidad}

\begin{consigna}{red}
	Para cada uno de los requerimientos del proyecto indique:
	\begin{itemize}
		\item Req \#1: copiar acá el requerimiento.

		      \begin{itemize}
			      \item Verificación para confirmar si se cumplió con lo requerido antes de mostrar el sistema al cliente. Detallar
			      \item Validación con el cliente para confirmar que está de acuerdo en que se cumplió con lo requerido. Detallar
		      \end{itemize}

	\end{itemize}

	Tener en cuenta que en este contexto se pueden mencionar simulaciones, cálculos, revisión de hojas de datos, consulta con expertos, mediciones, etc.  Las acciones de verificación suelen considerar al entregable como ``caja blanca'', es decir se conoce en profundidad su funcionamiento interno.  En cambio, las acciones de validación suelen considerar al entregable como ``caja negra'', es decir, que no se conocen los detalles de su funcionamiento interno.

\end{consigna}

\section{15. Procesos de cierre}
\label{sec:cierre}

\begin{consigna}{red}
	Establecer las pautas de trabajo para realizar una reunión final de evaluación del proyecto, tal que contemple las siguientes actividades:

	\begin{itemize}
		\item Pautas de trabajo que se seguirán para analizar si se respetó el Plan de Proyecto original:
		      - Indicar quién se ocupará de hacer esto y cuál será el procedimiento a aplicar.
		\item Identificación de las técnicas y procedimientos útiles e inútiles que se emplearon, y los problemas que surgieron y cómo se solucionaron:
		      - Indicar quién se ocupará de hacer esto y cuál será el procedimiento para dejar registro.
		\item Indicar quién organizará el acto de agradecimiento a todos los interesados, y en especial al equipo de trabajo y colaboradores:
		      - Indicar esto y quién financiará los gastos correspondientes.
	\end{itemize}

\end{consigna}


\end{document}
